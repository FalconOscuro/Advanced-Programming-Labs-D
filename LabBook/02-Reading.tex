\section{Reading}
    \subsection*{Question:}
        Implement the \mintinline{cpp}{Grid::LoadGrid(const char filename[])} method in Grid.cpp.
        This method should follow the following pseudo code:
        \begin{minted}{c}
            Create an input file stream from filename
            for each y value from 0 to 8 inclusive
            {
                for each x value from 0 to 8 inclusive
                {
                    store next value from the input file stream into grid at x,y
                }
            }
            Close input file stream
        \end{minted}
            
    \subsection*{Solution:}
        \begin{listing}[!ht]
            \inputminted[firstline=6, lastline=19]{cpp}{../Tasks/02-Reading/Grid.h}
            \caption{Grid.h}
        \end{listing}

        \newpage

        \begin{listing}[!ht]
            \inputminted[firstline=14]{cpp}{../Tasks/02-Reading/Grid.cpp}
            \caption{Grid.cpp}
        \end{listing}

    \subsection*{Test Data:}
        Data was input from the \textit{Grid1.txt} file.
        
    \subsection*{Sample Output:}

    \subsection*{Reflection:}
        I modified the base function template to better handle errors,
        this is done by returning a boolean for the success state.
        As for the function itself, a lot could be copied over from the previous
        lab.
            
        \begin{enumerate}
            \item The file is opened, using the given filename, 
                if the file cannot be opened then the function ends.

            \item The contents of the file are copied into a temporary array, 
                the program will loop through the file. Each iteration the program:
                \begin{itemize}
                    \item Check if the end of file has been reached, 
                        if so the function ends with a failure.

                    \item Copy the current number into the appropriate position in the temporary array.
                \end{itemize}
                A temporary array is used to avoid data being stored even on a failure.
                
            \item Finally the temporary array is copied into the class member and the success state is returned.
        \end{enumerate}

        Alongside this I also created an ovveride of the streaming operator for the grid
        class for ease of use when displaying the class.